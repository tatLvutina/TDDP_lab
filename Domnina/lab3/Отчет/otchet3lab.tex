\documentclass[russian,utf8,pointsection]{eskdtext}
\usepackage{eskdchngsheet}
\usepackage[T2A]{fontenc}
\usepackage{pscyr}
\usepackage{graphicx}
\usepackage{amstext}
\usepackage{amsmath}
\usepackage{listings}
\usepackage[unicode]{hyperref}
\usepackage{multirow}
\usepackage{listings}
\usepackage[utf8]{inputenc}
\usepackage[english,russian]{babel}

\graphicspath{{image/}}
\DeclareGraphicsExtensions{.pdf,.png,.jpg}


\ESKDdepartment{ГБОУ ВПО Нижегородский государственный технический университет им. Р. Е. Алексеева}
\ESKDcompany{Институт радиоэлектроники и информационных технологий, кафедра "Вычислительные системы и технологии"}
\ESKDtitle{Технологии распределённой обработки данных}
\ESKDdocName{Отчет к лабораторной работе №3}
\ESKDsignature{Распараллеливание алгоритма с помощью библиотеки CCR}
\ESKDauthor{Домнина~Н.~А.}
\ESKDtitleAgreedBy{Доцент каф. ВСТ}{Гай В. Е.}
\ESKDtitleDesignedBy{Студент гр. 13-В-2}{Домнина~Н.~А.}
\ESKDchecker{Гай В. Е.}
\ESKDdate{2015/11/17}

\begin{document}
	\maketitle
	\tableofcontents
	\newpage
	
	\section{Цель и порядок выполнения работы}
	  Цель работы: получить представления о возможности библиотеки Concurrent and Coordination Runtime для организации параллельных вычислений.
	
	Порядок выполнения работы:
	\begin{enumerate}
		\item Разработка последовательного алгоритма, решающего одну из приведённых задач в соответствии с выданным вариантом задания;
		\item Разработка параллельного алгоритма, соответствующий варианту последовательного алгоритма;
		\item Выполнение сравнения времени выполнения последовательного и параллельного алгоритмов обработки данных при различных размерностях исходных данных.
	\end{enumerate}
	\newpage 
	
	
	\lstset{
		breaklines=true, % Перенос длинных строк
		literate={а}{{\selectfont\char224}}1
		{б}{{\selectfont\char225}}1
		{в}{{\selectfont\char226}}1
		{г}{{\selectfont\char227}}1
		{д}{{\selectfont\char228}}1
		{е}{{\selectfont\char229}}1
		{ё}{{\"e}}1
		{ж}{{\selectfont\char230}}1
		{з}{{\selectfont\char231}}1
		{и}{{\selectfont\char232}}1
		{й}{{\selectfont\char233}}1
		{к}{{\selectfont\char234}}1
		{л}{{\selectfont\char235}}1
		{м}{{\selectfont\char236}}1
		{н}{{\selectfont\char237}}1
		{о}{{\selectfont\char238}}1
		{п}{{\selectfont\char239}}1
		{р}{{\selectfont\char240}}1
		{с}{{\selectfont\char241}}1
		{т}{{\selectfont\char242}}1
		{у}{{\selectfont\char243}}1
		{ф}{{\selectfont\char244}}1
		{х}{{\selectfont\char245}}1
		{ц}{{\selectfont\char246}}1
		{ч}{{\selectfont\char247}}1
		{ш}{{\selectfont\char248}}1
		{щ}{{\selectfont\char249}}1
		{ъ}{{\selectfont\char250}}1
		{ы}{{\selectfont\char251}}1
		{ь}{{\selectfont\char252}}1
		{э}{{\selectfont\char253}}1
		{ю}{{\selectfont\char254}}1
		{я}{{\selectfont\char255}}1
		{А}{{\selectfont\char192}}1
		{Б}{{\selectfont\char193}}1
		{В}{{\selectfont\char194}}1
		{Г}{{\selectfont\char195}}1
		{Д}{{\selectfont\char196}}1
		{Е}{{\selectfont\char197}}1
		{Ё}{{\"E}}1
		{Ж}{{\selectfont\char198}}1
		{З}{{\selectfont\char199}}1
		{И}{{\selectfont\char200}}1
		{Й}{{\selectfont\char201}}1
		{К}{{\selectfont\char202}}1
		{Л}{{\selectfont\char203}}1
		{М}{{\selectfont\char204}}1
		{Н}{{\selectfont\char205}}1
		{О}{{\selectfont\char206}}1
		{П}{{\selectfont\char207}}1
		{Р}{{\selectfont\char208}}1
		{С}{{\selectfont\char209}}1
		{Т}{{\selectfont\char210}}1
		{У}{{\selectfont\char211}}1
		{Ф}{{\selectfont\char212}}1
		{Х}{{\selectfont\char213}}1
		{Ц}{{\selectfont\char214}}1
		{Ч}{{\selectfont\char215}}1
		{Ш}{{\selectfont\char216}}1
		{Щ}{{\selectfont\char217}}1
		{Ъ}{{\selectfont\char218}}1
		{Ы}{{\selectfont\char219}}1
		{Ь}{{\selectfont\char220}}1
		{Э}{{\selectfont\char221}}1
		{Ю}{{\selectfont\char222}}1
		{Я}{{\selectfont\char223}}1
	}
	
	\lstset{
		language=Java,
		basicstyle=\footnotesize
	}
	
	
    \section{Теоретические сведения}
    \subsection{Библиотека Concurrent and Coordination Runtime}
    Библиотека Concurrent and Coordination Runtime (CCR) предназначена для организации обработки данных с помощью параллельно и асинхронно
    выполняющихся методов. Взаимодействие между такими методами организуется на основе сообщений. Рассылка сообщений основана на     использовании портов.
    Основные понятия CCR:
    \begin{enumerate}
    \item Сообщение – экземпляр любого типа данных;
    \item Порт – очередь сообщений типа FIFO (First-In-First-Out), сообщение остаётся в порте пока не будут извлечено из очереди порта     получателем.
    Определение порта:
    
    Port<int> p = new Port<int>();
    
    Отправка сообщения в порт:
    
    p.Post(1);
    
    \item получатель – структура, которая выполняет обработку сообщений.
    Данная структура объединяет:
    	\begin{itemize}
   \item один или несколько портов, в которые отправляются сообщения;
    \item метод (или методы), которые используются для обработки сообщений (такой метод называется задачей);
    \item логическое условие, определяющее ситуации, в которых активизируется тот или иной получатель.
    
    Делегат, входящий в получатель, выполнится, когда в порт intPort придёт сообщение.
    Получатели сообщений бывают двух типов: временные и постоянные (в примере получатель – временный). Временный получатель, обработав
    сообщение (или несколько сообщений), удаляется из списка получателей сообщений данного порта.
    \end{itemize}
        
    \item процессом запуска задач управляет диспетчер. После выполнения условий активации задачи (одним из условий активации может быть
    получение портом сообщения) диспетчер назначает задаче поток из пула потоков, в котором она будет выполняться.
    Описание диспетчера с двумя потоками в пуле:
    
    Dispatcher d = new Dispatcher(2, "MyPool");
    
    Описание очереди диспетчера, в которую задачи ставятся на выполнение:
    
    DispatcherQueue dq = new DispatcherQueue("MyQueue", d);
    
    \end{enumerate}
    
       \subsection{Создание проекта}
       Нужно выполнить следующие действия:
       \begin{enumerate}
       \item Установить библиотеку CCR (CCR входит в состав Microsoft Robotics Developer Studio);
       \item Создать проект консольного приложения и добавьте к проекту библиотеку Microsoft.Ccr.Core.dll.
       	\end{enumerate}
       	
       	\subsection{Оценка времени выполнения}
       	Время выполнения вычислений будем определять с помощью класса
       	\begin{lstlisting}
       	Stopwatch:       	
       	Stopwatch sWatch = new Stopwatch();       	
       	sWatch.Start();       	
       	<выполняемый код>       	
       	sWatch.Stop();       	
       	Console.WriteLine(sWatch.ElapsedMilliseconds.ToString());
       	      	\end{lstlisting}
       	
       	\section{Выполнение лабораторной работы}
       		\subsection{Вариант задания}
       		Вариант 6:
       		\begin{itemize}
       			\item Разработка алгоритма поиска максимального и минимального значения	массива.
       		\end{itemize}
       	\subsection{Листинг программы}
       
     
  \lstinputlisting[language=Java]
   {Program.cs}
       	
       	\subsection{Результат работы программы}
       	Скриншот работы программы представлен на Рис.\ref{ris:1_1}.
       	\begin{figure}[!h]
       		\centering{\includegraphics[scale = 3]{1_1}}
       		\caption{}
       		\label{ris:1_1}
       	\end{figure}
       	
	\section{Вывод}
	В результате выполнения лабораторной работы мы получили представление о возможности библиотеки Concurrent and Coordination Runtime для организации параллельных вычислений.
	Мы выяснили, что скорость работы параллельного алгоритма превосходит скорость работы последовательного алгоритма более чем в 2 раза. Быстродействие параллельного алгоритма напрямую зависит от числа используемых ядер.
		
\end{document}